\documentclass[journal,12pt,twocolumn]{IEEEtran}

\usepackage{setspace}
\usepackage{gensymb}
\singlespacing
\usepackage[cmex10]{amsmath}

\usepackage{amsthm}

\usepackage{mathrsfs}
\usepackage{txfonts}
\usepackage{stfloats}
\usepackage{bm}
\usepackage{cite}
\usepackage{cases}
\usepackage{subfig}

\usepackage{longtable}
\usepackage{multirow}

\usepackage{enumitem}
\usepackage{mathtools}
\usepackage{steinmetz}
\usepackage{tikz}
\usepackage{circuitikz}
\usepackage{verbatim}
\usepackage{tfrupee}
\usepackage[breaklinks=true]{hyperref}
\usepackage{graphicx}
\usepackage{tkz-euclide}

\usetikzlibrary{calc,math}
\usepackage{listings}
    \usepackage{color}                                            %%
    \usepackage{array}                                            %%
    \usepackage{longtable}                                        %%
    \usepackage{calc}                                             %%
    \usepackage{multirow}                                         %%
    \usepackage{hhline}                                           %%
    \usepackage{ifthen}                                           %%
    \usepackage{lscape}     
\usepackage{multicol}
\usepackage{chngcntr}

\DeclareMathOperator*{\Res}{Res}

\renewcommand\thesection{\arabic{section}}
\renewcommand\thesubsection{\thesection.\arabic{subsection}}
\renewcommand\thesubsubsection{\thesubsection.\arabic{subsubsection}}

\renewcommand\thesectiondis{\arabic{section}}
\renewcommand\thesubsectiondis{\thesectiondis.\arabic{subsection}}
\renewcommand\thesubsubsectiondis{\thesubsectiondis.\arabic{subsubsection}}


\hyphenation{op-tical net-works semi-conduc-tor}
\def\inputGnumericTable{}                                 %%

\lstset{
%language=C,
frame=single, 
breaklines=true,
columns=fullflexible
}
\begin{document}


\newtheorem{theorem}{Theorem}[section]
\newtheorem{problem}{Problem}
\newtheorem{proposition}{Proposition}[section]
\newtheorem{lemma}{Lemma}[section]
\newtheorem{corollary}[theorem]{Corollary}
\newtheorem{example}{Example}[section]
\newtheorem{definition}[problem]{Definition}

\newcommand{\BEQA}{\begin{eqnarray}}
\newcommand{\EEQA}{\end{eqnarray}}
\newcommand{\define}{\stackrel{\triangle}{=}}
\bibliographystyle{IEEEtran}
\raggedbottom
\setlength{\parindent}{0pt}
\providecommand{\mbf}{\mathbf}
\providecommand{\pr}[1]{\ensuremath{\Pr\left(#1\right)}}
\providecommand{\qfunc}[1]{\ensuremath{Q\left(#1\right)}}
\providecommand{\sbrak}[1]{\ensuremath{{}\left[#1\right]}}
\providecommand{\lsbrak}[1]{\ensuremath{{}\left[#1\right.}}
\providecommand{\rsbrak}[1]{\ensuremath{{}\left.#1\right]}}
\providecommand{\brak}[1]{\ensuremath{\left(#1\right)}}
\providecommand{\lbrak}[1]{\ensuremath{\left(#1\right.}}
\providecommand{\rbrak}[1]{\ensuremath{\left.#1\right)}}
\providecommand{\cbrak}[1]{\ensuremath{\left\{#1\right\}}}
\providecommand{\lcbrak}[1]{\ensuremath{\left\{#1\right.}}
\providecommand{\rcbrak}[1]{\ensuremath{\left.#1\right\}}}
\theoremstyle{remark}
\newtheorem{rem}{Remark}
\newcommand{\sgn}{\mathop{\mathrm{sgn}}}
\providecommand{\abs}[1]{\left\vert#1\right\vert}
\providecommand{\res}[1]{\Res\displaylimits_{#1}} 
\providecommand{\norm}[1]{\left\lVert#1\right\rVert}
%\providecommand{\norm}[1]{\lVert#1\rVert}
\providecommand{\mtx}[1]{\mathbf{#1}}
\providecommand{\mean}[1]{E\left[ #1 \right]}
\providecommand{\fourier}{\overset{\mathcal{F}}{ \rightleftharpoons}}
%\providecommand{\hilbert}{\overset{\mathcal{H}}{ \rightleftharpoons}}
\providecommand{\system}{\overset{\mathcal{H}}{ \longleftrightarrow}}
	%\newcommand{\solution}[2]{\textbf{Solution:}{#1}}
\newcommand{\solution}{\noindent \textbf{Solution: }}
\newcommand{\cosec}{\,\text{cosec}\,}
\providecommand{\dec}[2]{\ensuremath{\overset{#1}{\underset{#2}{\gtrless}}}}
\newcommand{\myvec}[1]{\ensuremath{\begin{pmatrix}#1\end{pmatrix}}}
\newcommand{\mydet}[1]{\ensuremath{\begin{vmatrix}#1\end{vmatrix}}}
\numberwithin{equation}{subsection}
\makeatletter
\@addtoreset{figure}{problem}
\makeatother
\let\StandardTheFigure\thefigure
\let\vec\mathbf
\renewcommand{\thefigure}{\theproblem}
\def\putbox#1#2#3{\makebox[0in][l]{\makebox[#1][l]{}\raisebox{\baselineskip}[0in][0in]{\raisebox{#2}[0in][0in]{#3}}}}
     \def\rightbox#1{\makebox[0in][r]{#1}}
     \def\centbox#1{\makebox[0in]{#1}}
     \def\topbox#1{\raisebox{-\baselineskip}[0in][0in]{#1}}
     \def\midbox#1{\raisebox{-0.5\baselineskip}[0in][0in]{#1}}
\vspace{3cm}
\title{AI1103: Assignment 1}
\author{Tanmay Garg \\CS20BTECH11063 EE20BTECH11048}
\maketitle
\newpage
\bigskip
\renewcommand{\thefigure}{\theenumi}
\renewcommand{\thetable}{\theenumi}
Download all python codes from 
\begin{lstlisting}
https://github.com/tanmaygar/AI-Course/blob/main/Assignment1/codes/problem5_7.py
\end{lstlisting}
%
and latex-tikz codes from 
%
\begin{lstlisting}
https://github.com/tanmaygar/AI-Course/blob/main/Assignment1/Assignment1.tex
\end{lstlisting}

\section*{Problem Statement: }
An urn contains 25 balls of which 10 balls
bear a mark ’X’ and the remaining 15 bear a
mark ’Y’.
A ball is drawn at random from the
urn, its mark is noted down and it is replaced.
If 6 balls are drawn in this way, find the
probability that:
\begin{enumerate}
    \item all will bear 'X' mark.
    \item not more than 2 will bear 'Y' mark.
    \item at least one ball will bear 'Y' mark.
    \item the number of balls with 'X' mark and 'Y' mark will be equal.
\end{enumerate}

\section*{Solution:}
Let X be the number of balls which have 'X' mark on them\\
Using the expression of binomial distribution
\[P(X = r) = {n \choose r} p^r q^{n-r}\]
We have
\[n = 6\]
\[p = \frac{2}{5} = 0.4\] 
\[q = \frac{3}{5} = 0.6\] 

For \((i)\) we need to find \(P(X = 6)\)
\[P(X = 6) = {6 \choose 6} p^6 q^0 = \left(\frac{2}{5} \right)^6 = 0.004096\]

For \((ii)\) we need to find \(P(X \geq 4)\)
\[P(X \geq 4)  = \sum_{r = 4}^6 {6 \choose r} p^r q^{n-r}\]
\[= {6 \choose 4} \left(\frac{2}{5} \right)^4 \left(\frac{3}{5} \right)^2 + {6 \choose 5} \left(\frac{2}{5} \right)^5 \left(\frac{3}{5} \right) + {6 \choose 6} \left(\frac{2}{5} \right)^6  \]
\[=\frac{432}{3125} + \frac{576}{15625} + \frac{64}{15625}\]
\[= \frac{112}{625} = 0.1792\]
\\
For \((iii)\) we need to find \(P(X \leq 5)\)
\[P(X \leq 5) = \sum_{r = 0}^5 {6 \choose r} p^r q^{n - r}\]
\[= {6 \choose 0} \left(\frac{3}{5} \right)^6 + {6 \choose 1} \left(\frac{2}{5} \right)^1 \left(\frac{3}{5} \right)^5 + {6 \choose 2} \left(\frac{2}{5} \right)^2 \left(\frac{3}{5} \right)^4 +{6 \choose 3} \left(\frac{2}{5} \right)^3 \left(\frac{3}{5} \right)^3\]\[ + {6 \choose 4} \left(\frac{2}{5} \right)^4 \left(\frac{3}{5} \right)^2 + {6 \choose 5} \left(\frac{2}{5} \right)^5 \left(\frac{3}{5} \right)\]
\[= \frac{729}{15625} + \frac{2916}{15625} + \frac{972}{3125} + \frac{864}{3125} + \frac{432}{3125} + \frac{576}{15625}\]
\[= \frac{15561}{15625} = 0.995904\]
For \((iv)\) we need to find \(P(X = 3)\)
\[P(X = 3) = {6 \choose 3} p^3 q^3\]
\[={6 \choose 3} \left(\frac{2}{5} \right)^3 \left(\frac{3}{5} \right)^3 = 0.27648\]

\end{document}