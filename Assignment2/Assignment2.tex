\documentclass[journal,12pt,twocolumn]{IEEEtran}
\usepackage{amsthm}
\allowbreak
\usepackage{setspace}
\usepackage{gensymb}
\singlespacing
\usepackage[cmex10]{amsmath}
\usepackage{caption}
\usepackage{amsthm}

\DeclareUnicodeCharacter{2212}{-}
\usepackage{tikz}
\usepackage{pgfplots}

\usepackage{mathrsfs}
\usepackage{txfonts}
\usepackage{stfloats}
\usepackage{bm}
\usepackage{cite}
\usepackage{cases}
\usepackage{subfig}

\usepackage{longtable}
\usepackage{multirow}

\usepackage{enumitem}
\usepackage{mathtools}
\usepackage{steinmetz}
\usepackage{tikz}
\usepackage{circuitikz}
\usepackage{verbatim}
\usepackage{tfrupee}
\usepackage[breaklinks=true]{hyperref}
\usepackage{graphicx}
\usepackage{tkz-euclide}
\graphicspath{ {./images/} }
\usetikzlibrary{calc,math}
\usepackage{listings}
    \usepackage{color}                                            %%
    \usepackage{array}                                            %%
    \usepackage{longtable}                                        %%
    \usepackage{calc}                                             %%
    \usepackage{multirow}                                         %%
    \usepackage{hhline}                                           %%
    \usepackage{ifthen}                                           %%
    \usepackage{lscape}     
\usepackage{multicol}
\usepackage{chngcntr}

\DeclareMathOperator*{\Res}{Res}

\renewcommand\thesection{\arabic{section}}
\renewcommand\thesubsection{\thesection.\arabic{subsection}}
\renewcommand\thesubsubsection{\thesubsection.\arabic{subsubsection}}

\renewcommand\thesectiondis{\arabic{section}}
\renewcommand\thesubsectiondis{\thesectiondis.\arabic{subsection}}
\renewcommand\thesubsubsectiondis{\thesubsectiondis.\arabic{subsubsection}}


\hyphenation{op-tical net-works semi-conduc-tor}
\def\inputGnumericTable{}                                 %%

\lstset{
%language=C,
frame=single, 
breaklines=true,
columns=fullflexible
}
\begin{document}


\newtheorem{theorem}{Theorem}[section]
\newtheorem{problem}{Problem}
\newtheorem{proposition}{Proposition}[section]
\newtheorem{lemma}{Lemma}[section]
\newtheorem{corollary}[theorem]{Corollary}
\newtheorem{example}{Example}[section]
\newtheorem{definition}[problem]{Definition}

\newcommand{\BEQA}{\begin{eqnarray}}
\newcommand{\EEQA}{\end{eqnarray}}
\newcommand{\define}{\stackrel{\triangle}{=}}
\bibliographystyle{IEEEtran}
\raggedbottom
\setlength{\parindent}{0pt}
\providecommand{\mbf}{\mathbf}
\providecommand{\pr}[1]{\ensuremath{\Pr\left(#1\right)}}
\providecommand{\qfunc}[1]{\ensuremath{Q\left(#1\right)}}
\providecommand{\sbrak}[1]{\ensuremath{{}\left[#1\right]}}
\providecommand{\lsbrak}[1]{\ensuremath{{}\left[#1\right.}}
\providecommand{\rsbrak}[1]{\ensuremath{{}\left.#1\right]}}
\providecommand{\brak}[1]{\ensuremath{\left(#1\right)}}
\providecommand{\lbrak}[1]{\ensuremath{\left(#1\right.}}
\providecommand{\rbrak}[1]{\ensuremath{\left.#1\right)}}
\providecommand{\cbrak}[1]{\ensuremath{\left\{#1\right\}}}
\providecommand{\lcbrak}[1]{\ensuremath{\left\{#1\right.}}
\providecommand{\rcbrak}[1]{\ensuremath{\left.#1\right\}}}
\theoremstyle{remark}
\newtheorem{rem}{Remark}
\newcommand{\sgn}{\mathop{\mathrm{sgn}}}
\providecommand{\abs}[1]{$\left\vert#1\right\vert$}
\providecommand{\res}[1]{\Res\displaylimits_{#1}} 
\providecommand{\norm}[1]{$\left\lVert#1\right\rVert$}
%\providecommand{\norm}[1]{\lVert#1\rVert}
\providecommand{\mtx}[1]{\mathbf{#1}}
\providecommand{\mean}[1]{E$\left[ #1 \right]$}
\providecommand{\fourier}{\overset{\mathcal{F}}{ \rightleftharpoons}}
%\providecommand{\hilbert}{\overset{\mathcal{H}}{ \rightleftharpoons}}
\providecommand{\system}{\overset{\mathcal{H}}{ \longleftrightarrow}}
	%\newcommand{\solution}[2]{\textbf{Solution:}{#1}}
\newcommand{\solution}{\noindent \textbf{Solution: }}
\newcommand{\cosec}{\,\text{cosec}\,}
\providecommand{\dec}[2]{\ensuremath{\overset{#1}{\underset{#2}{\gtrless}}}}
\newcommand{\myvec}[1]{\ensuremath{\begin{pmatrix}#1\end{pmatrix}}}
\newcommand{\mydet}[1]{\ensuremath{\begin{vmatrix}#1\end{vmatrix}}}
\numberwithin{equation}{subsection}
\makeatletter
\@addtoreset{figure}{problem}
\makeatother
\let\StandardTheFigure\thefigure
\let\vec\mathbf
\renewcommand{\thefigure}{\theproblem}
\def\putbox#1#2#3{\makebox[0in][l]{\makebox[#1][l]{}\raisebox{\baselineskip}[0in][0in]{\raisebox{#2}[0in][0in]{#3}}}}
     \def\rightbox#1{\makebox[0in][r]{#1}}
     \def\centbox#1{\makebox[0in]{#1}}
     \def\topbox#1{\raisebox{-\baselineskip}[0in][0in]{#1}}
     \def\midbox#1{\raisebox{-0.5\baselineskip}[0in][0in]{#1}}
\vspace{3cm}
\title{AI1103: Assignment 2}
\author{Tanmay Garg \\CS20BTECH11063 EE20BTECH11048}
\maketitle
\newpage
\bigskip
\renewcommand{\thefigure}{\theenumi}
\renewcommand{\thetable}{\theenumi}
Download all python codes from 
\begin{lstlisting}
need to put
\end{lstlisting}
%
and latex-tikz codes from 
%
\begin{lstlisting}
need to put
\end{lstlisting}

\section*{Problem GATE-EC-2019-Q47: }
A random variable $X$ takes values −1 and +1 with probabilities 0.2 and 0.8, respectively.
It is transmitted across a channel which adds noise $N$, so that the random variable at the
channel output is $Y = X + N$. The noise $N$ is independent of $X$, and is uniformly
distributed over the interval [−2 , 2]. The receiver makes a decision
\[
\hat{X} = \begin{cases}
            -1, &\text{if}\quad Y \leq \theta \\
             +1, &\text{if}\quad Y \geq \theta\\
            \end{cases}
\]
where the threshold $\theta  \in [−1,1]$ is chosen so as to minimize the probability of error
\pr {\hat{X} \neq X}. The minimum probability of error, rounded off to 1 decimal place, is?
\section*{Solution:}
We know that 
\begin{align}
    X = {-1,+1}\\
    N \in [-2,2]\\
    Y = X + N\\
    P(X = -1) = 0.2\\
    P(X = +1) = 0.8
\end{align}
Since $N$ is uniformly distributed\\
$\therefore$ the probability distribution function of $N$ is:

\begin{tikzpicture}
\begin{axis}[
    axis lines = left,
    xlabel = $x$,
    ylabel = $y$,
]
\addplot [
    domain=-3:3, 
    samples=100, 
    color=red,
]
{0.25};
\end{axis}
\end{tikzpicture}
\\
\begin{align}
\therefore
    P(\theta < -1 + N) &= \int_{\theta + 1}^2 \frac{1}{4}dx\\
    &= \frac{1}{4}(1-\theta)\label{eq:0.0.7}
\end{align}
\begin{align}
\therefore
    P(\theta> N + 1) &= \int_{-2}^{\theta -1} \frac{1}{4}dx\\
    &= \frac{1}{4}(1+\theta)\label{eq:0.0.9}
\end{align}
The probability of error:
\begin{align}
\nonumber
    P_e = P(-1)\cdot {P(\theta < -1 + N)} \\ 
            \quad+ P(1)\cdot {P(\theta> N + 1)}\label{eq:0.0.10}
\end{align}
Substituting \eqref{eq:0.0.7} and \eqref{eq:0.0.9} in \eqref{eq:0.0.10}. We get:
\begin{align}
    P_e = 0.2\cdot \frac{1}{4}(1-\theta) + 0.8\cdot \frac{1}{4}(1+\theta)
\end{align}
Putting $\theta = -1$, we get
\begin{align}
    P_e = 0.1
\end{align}
    

\end{document}